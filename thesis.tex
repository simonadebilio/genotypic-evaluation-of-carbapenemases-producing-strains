\documentclass[11pt]{report}
\title{\textbf{Genotipic Evaluation of Carbapenemases Producing Strains Isolated from Different Biological Materials}}
\author{Simona Debilio}
\date{23 March 2017}

\usepackage{graphicx}
\usepackage[utf8x]{inputenc}
\usepackage{grffile}
\usepackage[margin=1in]{geometry}
\usepackage[english]{babel}
\setlength{\parindent}{0pt}

\begin{document}

\maketitle

\tableofcontents

\chapter{Introduction}
One of the most important discoveries in human history has been the identification of substances that were able to fight and defeat bacterial infections.
These substances, called “antibiotics”, had an extraordinary impact on the outcome of bacterial infections and, consequently, helped to extend life expectancy \cite{ventola2015antibiotic}. 

The first antibiotic was penicillin, discovered by Sir Alexander Fleming in 1928 (Figure 1) and following developed by chemical companies.

At first, antibiotics were used to treat serious infections in the 1940s, and they had similar positive outcomes worldwide \cite{Spellberg2014}.

Unfortunately, after many decades since the use of the first antibiotics on patients, bacterial infections began to spread again and, more importantly, became a threat again.
This renewed spread of bacterial infections is due to the emergence of bacterial strains that are resistant to most of present antibiotics \cite{ventola2015antibiotic}.

The main causes that have provoked/caused this antibiotic resistance crisis, which is occurring worldwide, are antibiotic overuse and misuse. 
In addition, there has been a lack in the development of new drugs by the pharmaceutical industry \cite{nature2013}. 
Overall, this situation has a huge impact on human health (Figure 2).

As epidemiological studies have shown there is a direct relationship between the use of antibiotics and the emergence and spread of resistant bacteria strains \cite{huttner2013antimicrobial}.

Resistance can arise spontaneously through mutation, but it can also be acquired from genetic elements inherited from other bacteria. 
The acquisition of mobile genetic elements (i.e. plasmids) is called “horizontal gene transfer”, and it allows not only the transfer among different members of the same species, but also among members of different species.
Furthermore, antibiotics cause the death of drug sensitive competitors, leaving only the resistant bacteria alive and able to reproduce themselves as the result of natural selection \cite{doi:10.1093/emph/eou024}.

One way we can limit the spread of antibiotic resistant strains is by minimising the natural selection for resistance genes.
This can be achieved by reducing the use of antibiotics, avoiding their use unless it is strictly necessary, and enhancing infection prevention (e.g. isolating infected patients, and improving the hygiene). 
Moreover, it is necessary to stop abusing antibiotics in agriculture \cite{Spellberg2014} \cite{doi:10.1093/emph/eou024}.
Following the EU conference called “The Microbial Threat”, which took place in 1998 in Copenhagen, antibiotic resistance became an official EU issue for the first time. 
Since 2001, the European Council has highlighted the importance of reinforcing epidemiological surveillance, and improve the supervision of laboratories, in addition to the need to create a  coordinated structure at national level, in order to prevent and control the spread of antibiotic resistances.
In the last few years Italy has seen the spread of Gram-negative bacteria, mainly belonging to the \emph{Klebsiella pneumoniae} species ascribed to the Enterobacteriaceae family, that have become resistant to carbapenems (e.g. Imipenem and Meropenem).

A dramatically increasing trend has been observed: while in 2009 only the 1,3$\%$ of K. pneumoniae strains isolated from blood showed a resistance, the percentage rose to 16$\%$ in 2010, and to 26.7$\%$ in 2011.
These studies have confirmed the recent spread of multiresistant Enterobacteria in Italy, and they have shown why these bacteria represent a concrete threat for public health, as they are frequently the cause of infections, both in hospital and community environment.
(Ministero della Salute, dipartimento della sanità pubblica e dell’innovazione).

The role played by the laboratories of clinical microbiology is as essential as it is demanding: in order to monitor the spread of carbapenemase-producing Enterobacteria (CPE) and to realise appropriate measures to contain their spread, the medical personnel need to become familiar with a number of technical procedures for the rapid identification of resistant bacterial strains. 
Since this task can have a significant clinical impact on patients, the microbiologists need to undergo a constant methodological, technical and organizational training. 
Furthermore, the timely identification of the antibiotic-resistant strains allows to implement the effectiveness of infection control, through the application of risk mitigation measures.


\section{What is an infectious disease and how it is transmitted}

An infectious disease is a pathology caused by microbiological agents that, when coming into contact with a host, are able to reproduce themselves and cause a functional impairment: the disease is the result of the complex interaction between the host immune system and the foreign organism. 
The microorganisms causing infectious diseases are usually bacteria, viruses and fungi. 
Since these microorganisms need to take advantage of some vital functions of the host to replicate and survive, the relationship between them is identified as parasitism.
The human body defends itself from these attacks in many ways: the first barrier is represented by the presence of skin and mucous. They are able to resist to penetration of microorganisms thanks to their antimicrobial action, which is partly mechanical (e.g. tears, saliva, and urine), and partly chemical-physical (e.g. low pH). 
The time elapsing from the contact between the microorganism and its human host, and the onset of symptoms is said “period of incubation”, which varies depending on the disease and on the relationships that are established between the germ and its host. 
This period is defined “infection”, and consists of the presence of microbiological agents that reproduce inside the host body.
If symptoms appear during the infection, we have the onset of a “disease”, but the infection can also run without symptoms and, in that case, it is said to be an “asymptomatic infection”.
Contagious infectious diseases are caused by pathogens that are transmitted to receptive subjects. 
Infectious disease which are not contagious require instead the intervention of suitable vectors and specific circumstances. 
An effective prevention of infectious diseases can be achieved by removing one of their two causes: exposition to the pathogen and state of susceptibility.
(Le malattie infettive - Epicentro - Istituto Superiore di Sanità)

\chapter{What is an antibiotic and how it works}
Antibiotics are secondary metabolites produced naturally by bacteria and other soil microorganisms. 
In natural environment they probably play a defensive function for the organisms producing them. 
From a pharmacological point of view, they have revolutionised medicine, providing a “universal” cure for infectious diseases that had remained untreatable for centuries. 
Hundreds of molecules with antibacterial activity have been identified, and they have been divided into different classes on the base of the different chemical characteristics that distinguish the pharmacologically active molecule. 
Although every antibiotic interferes at some level with the survival of the bacteria, their mechanism of action can be very different. 
Antibiotics can alter the structure of the bacterial cell wall or cell membrane, with the energy metabolism, the synthesis of nucleic acids, or the protein synthesis.

Depending on the kind of antibiotic, there can be bactericidal effects, which directly cause the death of microorganisms, or bacteriostatic effects, which cause the inhibition of the bacterial reproduction, but not their death \cite{Leekha2011}.

Each antibiotic has its own spectrum of activity: some are effective on Gram-positive bacteria (e.g. penicillin), while others are effective on Gram-negative bacteria (e.g. cephalosporin). 
Even though there are broad-spectrum antibiotics, which are able to remove many of the Gram-positive and Gram-negative bacteria, no antibiotic is active against all bacteria. 
For this reason, it is essential to know the antibiotic range of action in order to obtain a targeted and effective pharmacological therapy.
Antibiotics distinguish themselves from other substances of microbial origin (e.g. toxins) because they can present high selectivity with regard to bacterial cells and low toxicity to eukaryotic cells. 
This mechanism, which is called “selective toxicity”, allows them to eliminate the infection without damaging the patient.

\section{$\beta$-lactam antibiotics}
$\beta$-lactam antibiotics are a class of broad-spectrum antibiotics which consists of all the antibiotics containing a $\beta$-lactam ring in their molecular structure, including penicillins, cephalosporins and carbapenems \cite{Pitout2005}. 
These antibiotics present a nucleus (6-aminopenicillanic acid) connected to different lateral chains that affect their pharmacokinetic peculiarities and their range of action.
Penicillins have a bicyclic structure, and the $\beta$-lactam ring is the functional part of the molecule: if degraded the drug loses its effectiveness.
$\beta$-lactam antibiotics are active against many Gram-negative and Gram-positive bacteria.

The antibacterial mechanism of penicillins occurs via the inhibition of the synthesis of the bacterial cell wall. If bacteria had no cell wall, their cells would rupture due to the difference in osmolarity between the inside and the outside of the cell.
The substance that confers resistance and rigidity to the cell wall is the peptidoglycan. This compound forms a mesh-like layer, consisting of glycosaminoglycan chains interlinked with short peptides. 
The sugar component consists of alternating disaccharide of N-acetylglucosamine (NAG) and N-acetylmuramic acid (NAM), connected by a $\beta$(1,4) glycosidic bond. There is a chain of four amino acids (both D- and L- ) attached to the N-acetylmuramic acid.

(immagine)

The structure is repetitive and resistant due to the third amino acid bond to the NAM of a NAM-NAG chain, which is tied to the fourth amino acid of a NAM on a parallel chain.
Penicillins block the formation of the cross-linking bonds within the peptidoglycan, therefore compromising its development in the cell wall. 
The development of these bonds is catalysed by a class of enzymes called “penicillin binding proteins” (PBP). These enzymes are able to remove one of the two residues of D-alanine placed at the end of the NAM pentapeptide.
$\beta$-lactam antibiotics are able to inhibit the PBP enzymes thanks to a competitive mechanism: since they are similar to the D-alanine D-alanine dimer, they are erroneously recognised by the enzyme as its substrates, causing the scission of the $\beta$-lactam antibiotic. 
The splitted antibiotic bonds covalently with the enzyme, forming an acyl-enzyme stable complex: this blocked enzyme is no longer capable of catalysing the peptidoglycan synthesis reactions, causing the death of the growing bacterial cells \cite{kong2010beta}.


\subsection{Carbapenem antibiotics}
A new era of $\beta$-lactam antibiotics, the carbapenems, has begun after the discovery of Streptomyces cattleya and its antibiotic product, the thienamycin.
After thienamycin, numerous carbapenems were discovered (e.g. imipenem, meropenem) \cite{Birnbaum1985}.
Like the penicillins, carbapenems are part of the $\beta$-lactam class of antibiotics, with a broader spectrum of activity; moreover their effectiveness is less affected by many of the most common mechanisms of antibiotic resistance.

(formula di struttura penicillina, carbapenemi)

This class of antibiotics presents a remarkable activity against both Gram-positive organisms and Enterobacteriaceae, Pseudomonas aeruginosa, and Bacteroides \cite{Neu1985}.

Since carbapenems are more effective against infections caused by multidrug-resistant bacteria than other $\beta$-lactam antibiotics, they are primarily used in hospitalised patients.

\chapter{The antibiotic resistance and the $\beta$-lactamases}

Antibiotics are considered one of the major breakthroughs of modern medicine. Their role has been essential in treating bacterial infections, and saving many lives. 
Unfortunately, time has seen the emergence and spread of antibiotic resistances among bacteria, weakening their effects. 
The development of combined resistances to multiple classes of antibiotics have brought to strains with multidrug-resistance (MDR) phenotypes, which can render traditional antibiotics completely ineffective \cite{Rossolini2014}.

Since $\beta$-lactams were the first antibiotics to be discovered, the resistance to this kind of antibiotics was the first to emerge, and to be understood. 
The most effective way for bacteria to react to these antibiotics has been by producing $\beta$-lactamases. 
These enzymes are able to hydrolyse the $\beta$-lactam ring of the antibiotics and, consequently, to inactivate them \cite{kong2010beta}.

Early works analysed and classified $\beta$-lactamases from a functional point of view. 
Today, their classification is based on the amino acid homology between different antibiotics, and has resulted in four different major classes: “molecular classes A, C, and D include the $\beta$-lactamases with serine at their active site, whereas molecular class B $\beta$-lactamases are all metallo-enzymes with an active-site zinc” \cite{Queenan2007}.

\section{The carbapenemases}
Among these four classes, carbapenemases are part of the classes A, B, and D.

\subsection{Class A carbapenemases}
Bacteria expressing class A serine carbapenemases have low susceptibility to imipenem, and “their MIC (Minimum Inhibitory Concentration) can range from mildly elevated to fully resistant”. 
This class is divided into three major families: NMC/IMI, SME and KPC enzymes.
All these carbapenemases are able to hydrolyse a broad variety of $\beta$-lactams (e.g. cephalosporins, penicillins, aztreonam) \cite{kong2010beta} \cite{Queenan2007}.

The first KPC-1 was discovered in a K. pneumoniae isolated  in North Carolina. The KPC family can spread easily thanks to its location on plasmids \cite{Queenan2007}.

This family is able to hydrolyse penicillins, cephalosporins, monobactams, carbapenems, and even $\beta$-lactamase inhibitors. 
Probably due to the few antibiotic options, there is a high mortality rate among patients infected with KPC positive bacteria \cite{MunozPrice2013}
                   
Although 10 different variations of KPC have been described, the KPC-2 and KPC-3 are the most commonly identified \cite{WaltherRasmussen2007}.

\section {Class B: metallo-$\beta$-lactamases}
This class of beta-lactamases is resistant to the available beta-lactam antibiotics, but it is inhibited by metal ion chelators.
They have a broad spectrum of activity: most of them are able to hydrolyse carbapenems, cephalosporins and penicillins.
Chromosomal enzymes (including VIM, IMP, GIM, and SIM) found in environmental and opportunistic pathogenic bacteria, were the first metallo-beta-lactamases to be detected and studied.
These enzymes were usually expressed in conjunction with at least one serine beta-lactamase \cite{Queenan2007}.

\section{Class D: OXA beta-lactamases}
This group of serine Beta-lactamases was firstly identified in Enterocateriaceae and Pseudomonas aeruginosa, and constitutes a separate class. 
The OXA beta-lactamases were described as penicillinases capable of hydrolysing oxacillin and cloxacillin. 
Out of the 102 unique OXA sequences which have currently been identified, 9 are extended spectrum beta-lactamases and more than 37 are considered to be carbapenemases.
Among this group, the OXA-48 variant is plasmid encoded, and presents the highest hydrolysis rate among all the OXA enzymes.
The OXA-48 was firstly found in a clinical K. pneumoniae isolated in Turkey \cite{Poirel2012}.

\chapter{Carbapenemases producing strains: Enterobacteriaceae}
The Enterobacteriaceae family includes a great variety of Gram-negative bacilli.
These bacilli can be aerobic or facultative anaerobic bacteria; most of them are mobile thanks to their flagella, but a few genera are nonmotile. 
They are not-forming spore bacteria, and typically $0,5-1,5\mu m$ in size.
This family includes both harmless symbiotic bacteria and pathogenic bacteria (such as Salmonella, Escherichia coli, and Klebsiella).

\subsubsection{Pathogenicity of Enterobacteriaceae}

The pathogenicity of Enterobacteriaceae is strictly bound to the antigenic structure of their cell wall, which contains three main categories of antigens.
The K (capsular) antigen is a component of the polysaccharide capsule that surrounds the bacteria. 
The main task of this capsular structure is to avoid phagocytosis and, consequently, the activation of the complement.
The 0 (somatic) antigen is responsible for the common symptoms of bacterial infections (i.e. fever, activation of the complement cascade, shock).
The outer membrane of the cell contains lipopolysaccharide (LPS),and the lipid A portion is endotoxic, while the O (somatic) antigen is serotype specific \cite{guentzel1996escherichia}.

The H antigens are flagellar proteins, and they are known for their invasiveness, and their ability to facilitate the ascension of uropathogenic bacteria from the bladder into the kidneys \cite{wiles2008origins}.

Some organisms have flagella distributed on their cell surface (e.g. Escherichia coli) and can present the H antigens, while others, which are nonmotile and nonflagellates (e.g. Klebsiella), have no H antigens. 

One of the most common Enterobacteria that have been isolated in humans is Klebsiella. Some strains belonging to this \emph{genus}, in particular the Klebsiella pneumoniae species, are showing resistance to almost every antibiotic available, and are responsible for the largest number of nosocomial infections. 

\section{Klebsiella}
Klebsiella is a Gram-negative bacterium ascribed to the Enterobacteriaceae family. 
This microorganism has two common habitats: the environment (e.g. surface water and soil), and the mucosal surfaces of mammals, such as humans.

Klebsiella mainly attacks immunocompromised individuals, especially if hospitalised. It can be particularly troublesome when infecting premature infants (it is often involved in neonatal sepsis), and intensive care units \cite{podschun1998klebsiella}.
Among the Klebsiella species, Klebsiella pneumoniae is the most clinically relevant: it is responsible for 70$\%$ of human infections \cite{Pitout2015}.
The most common sites of colonization in humans are the gastrointestinal tract, and the nasopharynx. It can cause infections in the urinary tract, and also pneumonia \cite{Pitout2015, podschun1998klebsiella}.
In the last decades, controlling this kind of infection has been difficult due to the emergence and spread of drug-resistant strains of K. pneumoniae.
The resistance of K. pneumoniae to antibiotics (such as cephalosporins, fluoroquinoles, and trimethoprim-sulfamethoxazole) which are often used to treat this kind of infection, delays the start of a proper therapy. This causes an increase in morbidity and mortality among patients.

Since carbapenems are often the last line of effective therapy available for the treatment of infections caused by multidrug-resistant (MDR) K. pneumoniae, the emerging resistance to them is particularly concerning.
 \cite{Pitout2015}.

\subsection{Mechanisms of resistance to carbapenems} 

%ADD REFERENCES

K. pneumoniae resistances are due to different mechanisms.

The production of beta-lactamases, along with the occurrence of permeability defects, can lead to a reduced susceptibility to carbapenems.
These beta-lactamases enzymes can belong to the class A extende-spectrum beta-lactamases (ESBLs) or to the class C AmpC cephalosporinases.
There are also carbapenemases, belonging to the molecular classes A, B or D, that do not require additional permeability defects.

Among the beta-lactamases enzymes, there is the KPC-type, which was discovered for the first time in a sample of K. pneumoniae isolated in North Carolina.
Researchers have currently described more than twenty different KPC variants, and the most common are KPC-2 and KPC-3.

"K. pneumoniae ST258 with KPC-2 and KPC-3 has contributed significantly to the dissemination of KPC enzymes worldwide."
\cite{Pitout2015}.

The ST258 clone is the most common clone of KPC-producing K. pneumoniae.

"Several different KPC-containing plasmids have been identified in ST258 and these plasmids often contain various genes encoding nonsusceptibility to different antimicrobial drugs" \cite{Pitout2015}.

Another important group of beta-lactamases are the metallo-beta-lactamases (MBLs). Bacteria with MBLs are often resistant to penicillins, carbapenems, cephalosporins, and cephamycins but remain susceptible to monobactams. Moreover, they are inhibited by metal chelators (e.g. EDTA and dipicolinic acid).

OXA-48 beta-lactamase is the only class D carbapene-hydrolysing beta-lactamase isolated from K. pneumoniae. This enzyme hydrolyses beta-lactams such as penicillins, hydrolyses carbapenems, and cephalosporins.


\subsection{KPC epidemiology}

The first KPC-producing strain was isolated in 1996 in a hospital in North Carolina, and was followed by a report describing KPC-positive isolates from New York City hospitals.
KPC-positive bacteria can be isolated in urine, respiratory, blood, and wound samples. 
In Italy, the first KPC-positive K. pneumoniae was isolated in 2008 in Florence. The isolate presented a KPC-3 enzyme, with the corresponding gene located in transposon Tn4401.
In 2009, a second report showed two KPC-2 positive K. pneumoniae isolated in Rome.
Between 2009 and 2011, thanks to an active surveillance in two hospital in Padua, almost two hundred cases were identified.
\cite{MunozPrice2013}
The last ten years have shown a worldwide increasing spread of CPE strains.
This phenomenon has had particular relevance in countries such as the United States of America, Israel, Puerto Rico, Colombia, and Greece.
The spread of carbapenemases among different strains is probably due to the dissemination of mobile genetic elements that can transfer their resistance genes to other microorganisms.
One of the causes for the spread of KPC epidemic clones is patients being transferred between different hospitals, or different countries \cite{circolare2013}.

\subsection{Treatment of infections due to carbapenemases-producing K. pneumoniae}
Mortality rates due to K. pneumoniae infections are usually between 23 and 75$\%$ \cite{karaiskos2014multidrug}. 
None of the strategies using the currently available antibiotics is optimal to cure carbapenemase-producing K.pneumoniae infections, and single antibiotic therapies are usually ineffective.
Severe infections due to a carbapenemase-producing K.pneumoniae strain can justify the use of a combination therapy that includes colistin and a carbapenem, or an aminoglycoside.
Often, the only antibiotics that show in vitro activity are polymyxins (e.g., colistin or polymyxin B), tigecycline, fosfomycin, and sometimes selected aminoglycosides \cite{rodriguez2015diagnosis}.
Since carbapenemase-producing bacteria are often resistant to other antibiotic classes too (e.g fluoroquinolones and aminoglycosides), it is important to make susceptibility tests for antibiotics such as polymyxins (e.g. colistin), fosfomycin, tigecycline, and rifampin. These antibiotics can represent the last resort for treating such infections \cite{adams2009activity}.
Even if KPC-producing strains are usually resistant to all beta-lactam antibiotics, some of them still show a certain susceptibility to temocillin. 
Moreover, NDM,VIM, and IMP producers are susceptible to aztreonam, while OXA-48 producers should be tested to verify if they are susceptible to the expanded-spectrum cephalosporins \cite{girlich2009ctx}.
Since single antibiotic therapies have proved not to be very effective, combined therapies are often prescribed, as they can maximise bacterial killing (synergistic effect), while reducing bacterial resistances \cite{Pitout2015}.
The mortality rate appears to be significantly lower in patients that have undergone combination therapies \cite{tzouvelekis2014treating, zavascki2013combination}, and the best antibiotic associations result from the administration (?) of two molecules showing in vitro activities against carbapenemase-producing strains \cite{falagas2013antibiotic, tzouvelekis2014treating}. 
The colistin (polymyxin E) antibiotic has been discovered more than 60 years ago \cite{karaiskos2014multidrug, rodriguez2015diagnosis}.
This moleceule is often used in combination therapies, and is significantly effective against various carbapenemase-producing strains \cite{falagas2013antibiotic, temkin2014carbapenem}.
Even if it shows nephrotoxicity as a side effect, and a poor lung penetration, colistin is the most popular antibiotic for treating carbapenemase-producing K. pneumoniae infections \cite{karaiskos2014multidrug, rodriguez2015diagnosis}. 
Unfortunately, due to the increased use of this antibiotic, 
colistin-resistant K pneumoniae strains have already been reported \cite{mammina2012ongoing}.
Another antibiotic available in Europe is fosfomycin, commonly used in combination with tigecycline and colistin, in order to treat MDR bacteria infections \cite{pontikis2014outcomes}.
Gentamicin is still effective against some KPC and OXA-48
producers.
Despite the ability to produce carbapenemases, carbapenems can be used as an antibiotic option, when the MICs of carbapenems are $\le 8mg/l$, against carbapenemase-producing K. pneumoniae.
This therapy can be successful when combined with a second antibiotic, or when a prolonged intravenous infusion regimen is used \cite{tzouvelekis2014treating, daikos2014carbapenemase, tumbarello2012predictors}.
As studies lead with an animal model of infection (i.e. mouse pneumonia) have shown, a dual-carbapenem therapy (i.e. meropenem and ertapenem) could be effective \cite{wiskirchen2014vivo}.
Meropenem has been shown to retain its efficacy, whereas most probably ertapenem could act ad a ``suicide'' molecule for carbapenemase activity.

The efficacy of this therapy, which involves double-carbapenem, has been proved in human patients infected with KPC-producing strains \cite{giamarellou2013effectiveness}. 
Other useful beta-lactams are extended-spectrum cephalosporins, effective against OXA-48 producers without ESBLs \cite{mimoz2012broad}, and aztreonam, which can be an option for treating MBL producers infections \cite{nordmann2011emerging}.

\chapter{Materials and methods}

The emergence of carbapenem-resistances in enterobacteria strains represents a significant clinical problem, since carbapenem antibiotics are the reference drugs for treating infections caused by multiresistant enterobacteria.

The carbapenem resistant enterobacteriaceae (CRE), especially if carbapenemases producers, are a danger for public health for different reasons:
\begin{itemize}
\item infections due to enterobacteria strains are frequent in both hospital and community settings, and the spread of CRE infections makes treating patients difficult;
\item the mortality rate attributable to CRE infections usually is $20-30\%$ \textbf{(Carmeli et al., 2010)}, but it can reach up to $70\%$ during bacteremia \textbf{(Mouloudi et al., 2010)};
\item these microorganisms can easily spread across different patients, and their resistances can be transmitted to other microorganisms through plasmids.
\end{itemize}

Experiences in different hospitals and countries have shown that it is possible to reduce, or even eradicate the spread of these microorganisms.
This can be achieved thanks to aggressive measures of control in the sanitary environment, aimed at promptly detecting the presence of the infection and its hosts. After identifying the infection, it is necessary to adopt measures to contain its spread (i.e. isolation, hand hygiene, environmental decontamination) \textbf{(Carmeli et al., 2010; Gupta et al., 2011).}

Enterobacteriaceae can produce different carbapenemases, especially belonging to classes A and B. 
More rarely, the carbapenem resistance can be due to resistance mechanisms to beta-lactam antibiotics combined with a porin deficit, or to D class carbapenemases (e.g. OXA-48).

It is important to monitor the Enterobacteriaceae isolates with the lowest level of resistance.
Suspicion that the isolates are carbapenemase-producing Enterobacteriaceae should arise when the MIC of the carbapenem is higher than epidemiological cut-off (ECOFF) of the respective wild-type strains.
The ECOFF values define the top end of the wild-type distribution: microorganisms with MIC values higher that their ECOFF have most likely acquired some type of resistance.
In order to find more effective therapies, and avoid the spread of antibiotic resistances, screening tests are required.
Different phenotypic and genotypic analyses are used in order to identify the pathogens and their antibiotic resistances.

For this project, various techniques of on plate cultivation, and automated tools (i.e. Vitek2, and the GeneXpert system) have been used.

\section{Sowing on plate}
Initially the material is sown on a culture medium containing a concentration of nutrients suitable for bacterial growth.
Depending on the biological material, suitable solid terrains can be used. Specific bacteria can be grown on terrains containing specific substances:
\begin{itemize}
\item blood samples from central vessels and peripheral veins are sown on chocolate agar, McConkey agar, Columbia agar with 5$\%$ Sheep Blood (COL-S), and Sabourad;
\item bronchoalveolar lavage and sputum samples need to be diluted with Sputasol (1:1) and, if the sample is considered suitable, are sown on COL-S, Columbia CNA agar, McConkey agar, and Chocolate agar with Bacitracin;
\item oropharyngeal swab samples are sown on TSA-II and Sabourad;
\item nasal swab samples are sown on MRA agar;
\item rectal swab samples are sown on ESBL agar and McConkey agar with a disk of Meropenem antibiotic;
\item cavitary liquid samples (i.e. pleural, pericardial, peritoneal, and synovial) are sown on COL-S, Schaedler KV Agar with 5$\%$ Sheep Blood, and bottles for both anaerobic and aerobic growth.
\end{itemize}

In order to identify colonies of K. pneumoniae, the most suitable terrains are:
\begin{itemize}
\item McConkey agar. This is a selective and differential culture terrain for the identification of Gram-negative and enteric bacteria. Since it contains crystal violet and bile salts, it is able to inhibit the growth of  Gram-positive bacteria. 
\end{itemize}

%proof read until here

 Enteric bacteria that have the ability to ferment lactose can be detected using carbohydrate lactose, and the pH indicator neutral red.[2]

 Questo terreno contiene inoltre il lattosio come unica fonte di carboidrati ed il rosso neutro come colorante. I batteri che fermentano il lattosio, tra
cui Klebsiella pneumoniae, si presentano sotto forma di colonie rosse senza la precipitazione
dei sali biliari;

• l’ESBL è un terreno cromogeno assolutamente nuovo ed innovativo ideato specificamente per
lo screening degli Enterobatteri produttori di beta-Lattamasi a Spettro Esteso. L’isolamento e la
rilevazione su ESBL si basa su una ricca capacità nutritiva con una miscela di antibiotici che
agiscono da marcatori per questo meccanismo di resistenza. Questo terreno cromogeno per-
mette un’identificazione immediata e diretta degli Enterobatteri ritrovati più frequentemente
dopo 18-24 ore di incubazione:














\bibliography{references}
\bibliographystyle{unsrt}

\end{document}