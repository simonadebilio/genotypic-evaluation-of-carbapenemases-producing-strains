\documentclass[11pt]{report}
\title{\textbf{Genotipic Evaluation of Carbapenemases Producing Strains Isolated from Different Biological Materials}}
\author{Simona Debilio}
\date{23 March 2017}

\usepackage{graphicx}
\usepackage[utf8x]{inputenc}
\usepackage{grffile}
\usepackage[margin=1in]{geometry}
\usepackage[english]{babel}
\setlength{\parindent}{0pt}

\begin{document}

\maketitle

\tableofcontents

\chapter{Introduction}
One of the most important discoveries in human history has been the identification of substances that were able to fight and defeat bacterial infections.
These substances, called “antibiotics”, had an extraordinary impact on the outcome of bacterial infections and, consequently, helped to extend life expectancy \cite{ventola2015antibiotic}. 

The first antibiotic was penicillin, discovered by Sir Alexander Fleming in 1928 (Figure 1) and following developed by chemical companies.

At first, antibiotics were used to treat serious infections in the 1940s, and they had similar positive outcomes worldwide \cite{Spellberg2014}.

Unfortunately, after many decades since the use of the first antibiotics on patients, bacterial infections began to spread again and, more importantly, became a threat again.
This renewed spread of bacterial infections is due to the emergence of bacterial strains that are resistant to most of present antibiotics \cite{ventola2015antibiotic}.

The main causes that have provoked/caused this antibiotic resistance crisis, which is occurring worldwide, are antibiotic overuse and misuse. In addition, there has been a lack in the development of new drugs by the pharmaceutical industry \cite{nature2013}. Overall, this situation has a huge impact on human health (Figure 2).

As epidemiological studies have shown there is a direct relationship between the use of antibiotics and the emergence and spread of resistant bacteria strains.

Resistance can arise spontaneously through mutation, but it can also be acquired from genetic elements inherited from other bacteria. The acquisition of mobile genetic elements (i.e. plasmids) is called “horizontal gene transfer”, and it allows not only the transfer among different members of the same species, but also among members of different species.
Furthermore, antibiotics cause the death of drug sensitive competitors, leaving only the resistant bacteria alive and able to reproduce themselves as the result of natural selection \cite{doi:10.1093/emph/eou024}.

One way we can limit the spread of antibiotic resistant strains is by minimizing the natural selection for resistance genes.
This can be achieved by reducing the use of antibiotics, avoiding their use unless it is strictly necessary, and enhancing infection prevention (e.g. isolating infected patients, and improving the hygiene). Moreover, it is necessary to stop abusing antibiotics in agriculture \cite{Spellberg2014} \cite{doi:10.1093/emph/eou024}.
Following the EU conference called “The Microbial Threat”, which took place in 1998 in Copenhagen, antibiotic resistance became an official EU issue for the first time. Since 2001, the European Council has highlighted the importance of reinforcing epidemiological surveillance, and improve the supervision of laboratories, in addition to the need to create a  coordinated structure at national level, in order to prevent and control the spread of antibiotic resistances.
In the last few years Italy has seen the spread of Gram-negative bacteria, mainly belonging to the \emph{Klebsiella pneumoniae} species ascribed to the Enterobacteriaceae family, that have become resistant to carbapenems (e.g. Imipenem and Meropenem).

A dramatically increasing trend has been observed: while in 2009 only the 1,3$\%$ of K. pneumoniae strains isolated from blood showed a resistance, the percentage rose to 16$\%$ in 2010, and to 26.7$\%$ in 2011.
These studies have confirmed the recent spread of multiresistant Enterobacteria in Italy, and they have shown why these bacteria represent a concrete threat for public health, as they are frequently the cause of infections, both in hospital and community environment.
(Ministero della Salute, dipartimento della sanità pubblica e dell’innovazione).

The role played by the laboratories of clinical microbiology is as essential as it is demanding: in order to monitor the spread of carbapenemase-producing Enterobacteria (CPE) and to realise appropriate measures to contain their spread, the medical personnel need to become familiar with a number of technical procedures for the rapid identification of resistant bacterial strains. Since this task can have a significant clinical impact on patients, the microbiologists need to undergo a constant methodological, technical and organizational training. Furthermore, the timely identification of the antibiotic-resistant strains allows to implement the effectiveness of infection control, through the application of risk mitigation measures.


\section{What is an infectious disease and how it is transmitted}

An infectious disease is a pathology caused by microbiological agents that, when coming into contact with a host, are able to reproduce themselves and cause a functional impairment: the disease is the result of the complex interaction between the host immune system and the foreign organism. The microorganisms causing infectious diseases are usually bacteria, viruses and fungi. Since these microorganisms need to take advantage of some vital functions of the host to replicate and survive, the relationship between them is identified as parasitism.
The human body defends itself from these attacks in many ways: the first barrier is represented by the presence of skin and mucous. They are able to resist to penetration of microorganisms thanks to their antimicrobial action, which is partly mechanical (e.g. tears, saliva, and urine), and partly chemical-physical (e.g. low pH). The time elapsing from the contact between the microorganism and its human host, and the onset of symptoms is said “period of incubation”, which varies depending on the disease and on the relationships that are established between the germ and its host. This period is defined “infection”, and consists of the presence of microbiological agents that reproduce inside the host body.
If symptoms appear during the infection, we have the onset of a “disease”, but the infection can also run without symptoms and, in that case, it is said to be an “asymptomatic infection”.
Contagious infectious diseases are caused by pathogens that are transmitted to receptive subjects. Infectious disease which are not contagious require instead the intervention of suitable vectors and specific circumstances. An effective prevention of infectious diseases can be achieved by removing one of their two causes: exposition to the pathogen and state of susceptibility.
(Le malattie infettive - Epicentro - Istituto Superiore di Sanità)

\chapter{What is an antibiotic and how it works}
Antibiotics are secondary metabolites produced naturally by bacteria and other soil microorganisms. In natural environment they probably play a defensive function for the organisms producing them. From a pharmacological point of view, they have revolutionized medicine, providing a “universal” cure for infectious diseases that had remained untreatable for centuries. Hundreds of molecules with antibacterial activity have been identified, and they have been divided into different classes on the base of the different chemical characteristics that distinguish the pharmacologically active molecule. Although every antibiotic interferes at some level with the survival of the bacteria, their mechanism of action can be very different. Antibiotics can alter the structure of the bacterial cell wall or cell membrane, with the energy metabolism, the synthesis of nucleic acids, or the protein synthesis.


Depending on the kind of antibiotic, there can be bactericidal effects, which directly cause the death of microorganisms, or bacteriostatic effects, which cause the inhibition of the bacterial reproduction, but not their death \cite{Leekha2011}.


Each antibiotic has its own spectrum of activity: some are effective on Gram-positive bacteria (e.g. penicillin), while others are effective on Gram-negative bacteria (e.g. cephalosporin). Even though there are broad-spectrum antibiotics, which are able to remove many of the Gram-positive and Gram-negative bacteria, no antibiotic is active against all bacteria. For this reason, it is essential to know the antibiotic range of action in order to obtain a targeted and effective pharmacological therapy.
Antibiotics distinguish themselves from other substances of microbial origin (e.g. toxins) because they can present high selectivity with regard to bacterial cells and low toxicity to eukaryotic cells. This mechanism, which is called “selective toxicity”, allows them to eliminate the infection without damaging the patient.

\section{$\beta$-lactam antibiotics}
$\beta$-lactam antibiotics are a class of broad-spectrum antibiotics which consists of all the antibiotics containing a $\beta$-lactam ring in their molecular structure, including penicillins, cephalosporins and carbapenems \cite{Pitout2005}. 
These antibiotics present a nucleus (6-aminopenicillanic acid) connected to different lateral chains that affect their pharmacokinetic peculiarities and their range of action.
Penicillins have a bicyclic structure, and the $\beta$-lactam ring is the functional part of the molecule: if degraded the drug loses its effectiveness.
$\beta$-lactam antibiotics are active against many Gram-negative and Gram-positive bacteria.

The antibacterial mechanism of penicillins occurs via the inhibition of the synthesis of the bacterial cell wall. If bacteria had no cell wall, their cells would rupture due to the difference in osmolarity between the inside and the outside of the cell.
The substance that confers resistance and rigidity to the cell wall is the peptidoglycan. This compound forms a mesh-like layer, consisting of glycosaminoglycan chains interlinked with short peptides. The sugar component consists of alternating disaccharide of N-acetylglucosamine (NAG) and N-acetylmuramic acid (NAM), connected by a $\beta$(1,4) glycosidic bond. There is a chain of four amino acids (both D- and L- ) attached to the N-acetylmuramic acid.

(immagine)

The structure is repetitive and resistant due to the third amino acid bond to the NAM of a NAM-NAG chain, which is tied to the fourth amino acid of a NAM on a parallel chain.
Penicillins block the formation of the cross-linking bonds within the peptidoglycan, therefore compromising its development in the cell wall. The development of these bonds is catalysed by a class of enzymes called “penicillin binding proteins” (PBP). These enzymes are able to remove one of the two residues of D-alanine placed at the end of the NAM pentapeptide.
$\beta$-lactam antibiotics are able to inhibit the PBP enzymes thanks to a competitive mechanism: since they are similar to the D-alanine D-alanine dimer, they are erroneously recognized by the enzyme as its substrates, causing the scission of the $\beta$-lactam antibiotic. The splitted antibiotic bonds covalently with the enzyme, forming an acyl-enzyme stable complex: this blocked enzyme is no longer capable of catalyzing the peptidoglycan synthesis reactions, causing the death of the growing bacterial cells \cite{KONG2010}.


\subsection{Carbapenem antibiotics}
A new era of $\beta$-lactam antibiotics, the carbapenems, has begun after the discovery of Streptomyces cattleya and its antibiotic product, the thienamycin.
After thienamycin, numerous carbapenems were discovered (e.g. imipenem, meropenem) \cite{Birnbaum1985}.
Like the penicillins, carbapenems are part of the $\beta$-lactam class of antibiotics, with a broader spectrum of activity; moreover their effectiveness is less affected by many of the most common mechanisms of antibiotic resistance.

(formula di struttura penicillina, carbapenemi)

This class of antibiotics presents a remarkable activity against both Gram-positive organisms and Enterobacteriaceae, Pseudomonas aeruginosa, and Bacteroides \cite{Neu1985}.

Since carbapenems are more effective against infections caused by multidrug-resistant bacteria than other $\beta$-lactam antibiotics, they are primarily used in hospitalized patients.

\chapter{The antibiotic resistance and the $\beta$-lactamases}

Antibiotics are considered one of the major breakthroughs of modern medicine. Their role has been essential in treating bacterial infections, and saving many lives. Unfortunately, time has seen the emergence and spread of antibiotic resistances among bacteria, weakening their effects. The development of combined resistances to multiple classes of antibiotics have brought to strains with multidrug-resistance (MDR) phenotypes, which can render traditional antibiotics completely ineffective \cite{Rossolini2014}.


Since $\beta$-lactams were the first antibiotics to be discovered, the resistance to this kind of antibiotics was the first to emerge, and to be understood. The most effective way for bacteria to react to these antibiotics has been by producing $\beta$-lactamases. These enzymes are able to hydrolyse the $\beta$-lactam ring of the antibiotics and, consequently, to inactivate them \cite{KONG2010}.


In the early work with $\beta$-lactamases, they were analyzed and classified from a functional point of view. Today, the classification is based on the amino acid homology between different antibiotics, and has resulted in four different major classes: “molecular classes A, C, and D include the $\beta$-lactamases with serine at their active site, whereas molecular class B $\beta$-lactamases are all metallo-enzymes with an active-site zinc” \cite{Queenan2007}.


\section{The carbapenemases}
Among these four classes, carbapenemases are part of the classes A, B, and D.

\subsection{Class A carbapenemases}
Bacteria expressing class A serine carbapenemases have low susceptibility to imipenem, and “their MIC (Minimum Inhibitory Concentration) can range from mildly elevated to fully resistant”. This class is divided into three major families: NMC/IMI, SME and KPC enzymes.
All these carbapenemases are able to hydrolyse a broad variety of $\beta$-lactams (e.g. cephalosporins, penicillins, aztreonam) \cite{KONG2010} \cite{Queenan2007}.


The first KPC-1 was discovered in a K. pneumoniae isolated  in North Carolina. The KPC family can spread easily thanks to its location on plasmids \cite{Queenan2007}.

This family is able to hydrolyse penicillins, cephalosporins, monobactams, carbapenems, and even $\beta$-lactamase inhibitors. Probably due to the few antibiotic options, there is a high mortality rate among patients infected with KPC positive bacteria \cite{MunozPrice2013}
                   
Although 10 different variations of KPC have been described, the KPC-2 and KPC-3 are the most commonly identified \cite{WaltherRasmussen2007}.



\section {Class B: metallo-$\beta$-lactamases}
This class of beta-lactamases is resistant to the available beta-lactamase antibiotic, but they are inhibited by metal ion chelators.
They have quite a broad spectrum of activity: most of them are able to hydrolyze carbapenems, cephalosporins and penicillins.
“The first metallo-beta-lactamases detected and studied were chromosomal enzymes present in environmental and opportunistic pathogenic bacteria”.
These enzymes where usually expressed in conjunction with at least one serine beta-lactamase.
“The most common metallo-beta-lactamase families include the VIM, IMP, GIM, and SIM enzymes.”
\cite{Queenan2007}

\section{Class D: OXA beta-lactamases}
This class of Beta-lactamases was firstly identified in the Enterocateriaceae and P. aeruginosa.
Even if they form (?) a separate class they are serine beta-lactamases too .
The OXA beta-lactamases were described as penicillinases capable of hydrolyzing oxacillin and cloxacillin.
“Currently there have been 102 unique OXA sequences identified, of which 9 are extended spectrum beta-lactamases and at least 37 are considered to be carbapenemases”.
Among this group, the OXA-48 variant is plasmid encoded, and presents the highest hydrolysis rate among all the OXA enzymes.
The OXA-48 was firstly found in a clinical K. pneumoniae isolate from Turkey \cite{Poirel2012}.


\chapter{Carbapenemases producing strains: the Enterobacteriaceae}
The Enterobacteriaceae family includes a great variety of Gram-negative bacilli. These bacilli can be aerobic or facultative anaerobic bacteria, most of them are mobile thanks to their flagella, but a few genera are nonmotile. They are not-forming spore bacteria, and are typically 0,5-1,5 micron 
This family includes not only pathogenic bacteria, such as Salmonella, Escherichia coli, and Klebsiella, but also many harmless symbiotic bacteria.

\subsubsection{Enterobacteriaceae pathogenicity}

The Enterobacteriaceae pathogenicity is strictly bound to the antigenic structure of their cell wall, which contains three main categories of antigens.
The K (capsular) antigen is a component of the polysaccharide capsuler that surrounds the bacteria. Main task of this capsular structure is to avoid phagocytosis and, consequently, the complement activation. 
\textbf{(Chapter 26Escherichia, Klebsiella, Enterobacter, Serratia, Citrobacter, and Proteus)}
The 0 (somatic) antigen is responsible for the common symptom of bacterial infections (i.e. fever, activation of the complement cascade, shock).
“The outer membrane contains lipopolysaccharide (LPS), of which the lipid A portion is endotoxic and the O (somatic) antigen is serotype specific”.  
\textbf{(Chapter 26 Escherichia, Klebsiella, Enterobacter, Serratia, Citrobacter, and Proteus)}

The H (flagellar) antigens are flagellar proteins. They are important for the invasiveness, and ability to go up the renal parenchyma, of the uropathogenic species.

Some organisms have flagella distributed on their cell surface (e.g. Escherichia coli) and can present the H antigens, while others, which are nonmotile and nonflagellates (e.g. Klebsiella), have no H antigens. 

One of the most common Enterobacteria isolated from human patients (NON CREDO PROPRIO VADA BENE) is Klebsiella. Some strains belonging to this genus (?), especially the Klebsiella pneumoniae species, are resistant to almost every available antibiotic.

\section{The Klebsiella}
Klebsiella is a Gram-negative bacteria ascribed to the Enterobacteriaceae family. 
This microorganism has two common habitats: the environment (e.g. surface water and soil), and the mucosal surfaces of mammals such as humans.

Klebsiella principally/mainly attacks immunocompromised individuals, especially if hospitalized. It can be particularly troublesome in premature infants (its often involved in neonatal sepsis), and intensive care units. 
The medically most important species of this genus is Klebsiella pneumoniae, which is responsible for the largest number of nosocomial infections. 
\textbf{Klebsiella spp. as Nosocomial Pathogens: Epidemiology, Taxonomy,
Typing Methods, and Pathogenicity Factors}

Among the Klebsiella species, Klebsiella pneumoniae is the most clinically relevant: it is responsible for 70$\%$ of human infections. \textbf{Carbapenemase-Producing Klebsiella pneumoniae, a Key Pathogen Set for Global Nosocomial Dominance}

The most common sites of colonization in humans are the gastrointestinal tract, and nasopharynx. It can cause infections in the urinary tract, and also pneumonia. \textbf{Carbapenemase-Producing Klebsiella pneumoniae, a Key Pathogen Set
for Global Nosocomial Dominance AND Klebsiella spp. as Nosocomial Pathogens: Epidemiology, Taxonomy,Typing Methods, and Pathogenicity Factors }













The management of infections due to K. pneumoniae has been
complicated by the emergence of antimicrobial resistance, espe-
cially since the 1980s. The cephalosporins, fluoroquinolones, and
trimethoprim-sulfamethoxazole are often used to treat infections
due to K. pneumoniae, and resistance to these agents generates
delays in appropriate empirical therapy with subsequent increased
morbidity and mortality in patients

Of special concern is the emerging resistance to carbapen-
ems, since these agents are often the last line of effective therapy
available for the treatment of infections caused by multidrug-re-
sistant (MDR) K. pneumoniae















































\bibliography{references}
\bibliographystyle{unsrt}

\end{document}